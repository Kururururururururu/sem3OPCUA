\section{Conclusion}
Based on the analysis we came to the conclusion that our target group is hobbyists or home brewers.
We have also concluded that the most important features for our target group is the ability to simplify the brewing process for the users and
to make it easier and less time consuming, since the brewing process is very complex and time consuming
and we don't expect our users to have any prior experience or expertise in brewing.\newline

With the help of the analysis, requirements for the project was made.
Based on these requirements we made a design for the system.
In the design we decided to use a client-server architecture, where the client is a web application and the server is a REST API.
The REST API is used to communicate with the database and the OPC-UA server.
The OPC-UA server is used to communicate with the machine, where the brewing process takes place.
We also made a database for storing the order with the beer type, the amount of beers and the timestamp.\newline

The implementation of the system was made with the design in mind.
In the web application we decided to use a single page application (SPA) with the UI library React,
since is makes it easier to update the UI without having to reload the page.
We decided to go for a backend with Node.js instead of Java, because it is easier to use and faster to develop with than Java.
The Milo OPC-UA library in Java is also using Java 8, which is a very old version.
The REST API was made with the framework Express and the OPC-UA client was made with the library node-opcua.
The database was made with the database management system (DBMS) PostgreSQL, using the pg library.\newline

To make sure our system worked as we intended, we tested the system.
We made unit tests for the Express REST API backend and for the OPC-UA client.
We also tested each of the beer types with different speeds and observed the fail rate,
as one of our requirements was to aim for optimal speed for each beer type.
From the data gathered, graphs were made for each beer type together with a regression line.
Based on that regression line we solved for having a failrate of 1\%.
From the solution, the optimal speeds for each beer type was 263, 5, 58, 238, -10 and -198 for Pilsner, Wheat, IPA, Stout, Ale and Alcohol Free respectively.
As negative speeds aren't possible we decided to use 1 as the speed for Ale and Alcohol Free.\newline