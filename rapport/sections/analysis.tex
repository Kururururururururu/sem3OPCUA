
\section{Case-analysis}
In our project we have carefully considered the target group and adapted our vision for the project to comply with the stakeholders' needs.
We have based our plan on the fact that our target group primarly consists of people who have a passion for 
producing beer in small quantities. We imagine that it could be intended for private consumption. The target group is described further in the personas section.
To accommodate the stakeholders' expectations we have decided to develop a user-friendly web interface. We define a user-friendly web interface as being very easy to navigate and use.
In addition to the user interface being practical, we also want our program to be reliable and to be able to run a 
beer production without major losses. This has to be done without needing a lot of input from the user. Thus, making it mostly automatic.

In our project we have relied on the knowledge we have gained through the various courses we have had this semester.
This knowledge has impacted our project differently. We have for example identified the applicability of the knowledge gained in 'Industrial Cyber-Physical Systems' when using OPC UA in the 
interaction between the web interface and the beer production machine. Additionally, 'Calculus and Linear Algebra' has contributed to our ability to calculate the optimal production speed for our machine.
This allowed us to maintain a satisfactory error margin. Moreover, the course informed our decision to plot and, thereby, visually present our data.
Contrastingly, utilizing the knowledge from 'Operating Systems and Distributed Systems' allowed us to develop a Docker Container, which was beneficial in relation to the database.
It was also adventageous for our project to apply our experiences from 'Web Technologies'. Specifically, the course laid the foundation for us to build the web interface as we for example relied on 
our experiences with HTML, CSS, and JavaScript.



\subsection{Personas}
To define the target group for the project and to help ourselves understand the project's possibilities and functionalities, 
we have decided to make some personas. These characters are fictitious and based on the project's stakeholders. \newline

\begin{table}[htb]
    \begin{center}
        \begin{tabular}{|p{16cm}|}
            \hline
            \textbf{Kim, beer hobbyist}                                                                                                                                                                                                                        \\
            \hline
            \\ \textbf{Bio:} Kim is 42 years of age and he has a big passion for beers. It has been a big tradition in his family to brew their own beer. He recently decided that he wanted his own brewing machine to help him brew beer. He is not experienced in computers, so it must be user-friendly.  \\
            \\
            \textbf{Goal:} Kim is very excited to get his new brewing machine for his garage. He does have some concerns when it comes to the software part. He hopes that the brewing machine is easy to operate so that he can make beers whenever he wants to. \\
            \hline
        \end{tabular}
        \caption{Persona Kim}
        \label{tab:persona_kim}
    \end{center}
\end{table}

\begin{table}[htb]
    \begin{center}
        \begin{tabular}{|p{16cm}|}
            \hline
            \textbf{Lars, busy lawyer}                                                                                                                                           \\
            \hline
            \\ \textbf{Bio:} Lars is 61 years of age and works as a lawyer. He always works and don't have a lot of time to spare. He has a passion for beers and would love to brew his own. His time is limited so the process of brewing should be simple and efficient.  \\
            \\
            \textbf{Goal:} Lars wants to create quality beer without too much effort in the brewing process. He needs a machine that is user-friendly while also being effective. \\
            \hline
        \end{tabular}
        \caption{Persona Lars}
        \label{tab:persona_lars}
    \end{center}
\end{table}

\begin{table}[htb]
    \begin{center}
        \begin{tabular}{|p{16cm}|}
            \hline
            \textbf{Anna, Automation Enthusiast}                                                                                                                                                              \\
            \hline
            \\ \textbf{Bio:} Anna is 28 years of age and is an automation nerd. She has always been impressed by machines that are automated. She is passionate about beers and has brewed some before manually. However, recently she has had interest in getting an automatic brewing machine.  \\
            \\
            \textbf{Goal:} Anna aims to achieve tasty beers with only one click, which selects the beer she wants brewed.\\
            \hline
        \end{tabular}
        \caption{Persona Anna}
        \label{tab:persona_anna}
    \end{center}
\end{table}

\subsection{User Stories}

We have decided to create some 'user stories' which together with the 'personas' lay the foundation for the development of requirements.
The 'user stories' enable us to describe the functionality of a system from an end user's perspective.

US-8, US-10

\begin{table}[H]
    \begin{center}
        \sloppy
        \begin{longtable}{|p{1cm}|p{11cm}|c|c|}
            \hline
            ID    & Stories                                                                                                                                    & Estimation & Moscow \\ \hline
            US-1  & As a user I want the interface to be user-friendly, so that I easily navigate it                                                           & 9          & M      \\ \hline
            US-2  & As a user I want some already defined beers so that I can select the beer I want without too much effort                                   & 8          & M      \\ \hline
            US-3  & As a user I want the brewing process to be automated so that when I click on a specific button it will brew the corresponding beer         & 7          & M      \\ \hline
            US-4  & As a user I want the machine to automatically handle difficulties such as maintenance so that I don't have to do anything                  & 8          & M      \\ \hline
            US-5  & As a user I want to get displayed different data while the machine is brewing so I can keep track of the machine's status                  & 5          & C      \\ \hline
            US-6  & As a user I want the machine to be reliable so that I am guaranteed that the beers are produced with only limited issues                   & 6          & S      \\ \hline
            US-7  & As a user I want the machine to be efficient so that I don't have to wait too long to get a beer                                           & 5          & C      \\ \hline
            US-8  & As a user I want the machine to be scalable so that I can add new beers to the machine                                                     & 4          & C      \\ \hline
            US-9  & As a user I want the machine to have a quick reponse time so that I am ensured that the machine correctly understands the input            & 3          & W      \\ \hline
            US-10 & As a user I want the machine to be able to handle different amount of beers so that I can choose the amount I want                         & 3          & W      \\ \hline

            \caption{User stories for the beer machine}
            \label{tab:user_stories}
        \end{longtable}
    \end{center}
\end{table}