
\section{Analysis}

In this project we have worked on creating a web interface that can be used to interact with a beer production 
machine. In order to ensure this interaction between the beer machine and our website, we have used OPC UA.\newline
Our plan with this web interface is for it to be user-friendly and therefore very easy to navigate and use. 
In addition to the user interface being practical, we also want our program to be reliable and to be able to run a 
beer production without major losses. This has to be done without needing a lot of input from the user, making 
it mostly automatic.\newline
We have based our plan on the fact that our target group primarly consists of people who have a passion for or a hobby of 
producing beer in small quantities. We imagine that it could be intended for own consumption. Further described in the personas section.\newline

\subsection{Personas}
To define the target group for the project and to help ourselves understand the project's possibilities and functionalities, 
we have decided to make some personas. These characters are fictitious and based on the project's stakeholder. \newline

\begin{table}[htb]
    \begin{center}
        \begin{tabular}{|p{16cm}|}
            \hline
            \textbf{Kim, beer hobbyist}                                                                                                                                                                                                                        \\
            \hline
            \\ \textbf{Bio:} Kim is 42 years of age, he has a big passion for beers. It has been a big tradition in his family to brew their own beer. He recently decided that he wanted his own brewing machine to help him brew beer. He is not experienced in computers, so it must be user-friendly.  \\
            \\
            \textbf{Goal:} Kim is very excited to get his new brewing machine for his garage. He does have some concerns when it comes to the software part. He hopes that the brewing machine is easy to operate so that he can make beers when he wants too. \\
            \hline
        \end{tabular}
        \caption{Persona Kim}
        \label{tab:persona_kim}
    \end{center}
\end{table}

\begin{table}[htb]
    \begin{center}
        \begin{tabular}{|p{16cm}|}
            \hline
            \textbf{Lars, busy lawyer}                                                                                                                                           \\
            \hline
            \\ \textbf{Bio:} Lars is 61 years of age and works as a lawyer. He always works and don't have a lot of time to spare. He has a passion for beers and would love to brew his own. His time is limited so the process of brewing should be simple and efficient.  \\
            \\
            \textbf{Goal:} Lars want to create quality beer, without too much effort in the brewing process. He needs a machine that is user-friendly while also being effective. \\
            \hline
        \end{tabular}
        \caption{Persona Lars}
        \label{tab:persona_lars}
    \end{center}
\end{table}

\begin{table}[htb]
    \begin{center}
        \begin{tabular}{|p{16cm}|}
            \hline
            \textbf{Anna, Automation Enthusiast}                                                                                                                                                              \\
            \hline
            \\ \textbf{Bio:} Anna, 28, is an automation nerd. She has always been impressed by machines that are automated. She is passionate about beers, and have brewed some before manually, but recently found interest in getting an automatic brewing machine for beers.  \\
            \\
            \textbf{Goal:} Anna aims to achieve tasty beers with only one click, which is selecting the beer she wants brewed with the new machine, she can get that, while also having an automated machine. \\
            \hline
        \end{tabular}
        \caption{Persona Anna}
        \label{tab:persona_anna}
    \end{center}
\end{table}

\subsection{User Stories}

With the help of our personas and our requirements, we have decided to create some 'user stories' because they can help 
describe the functionality of a system from an end user's perspective.

\begin{table}[H]
    \begin{center}
        \sloppy
        \begin{longtable}{|p{1cm}|p{11cm}|c|c|}
            \hline
            ID    & Stories                                                                                                                       & Estimation & Moscow \\ \hline
            US-1  & As a user I want the interface to be user-friendly, so that everyone can use it                                               & 9          & M      \\ \hline
            US-2  & As a user I want some already defined beers so that I can select the beer I want without too much thought                     & 8          & M      \\ \hline
            US-3  & As a user I want the brewing process to be automated so that I can click on a beer, it will start brewing                     & 7          & M      \\ \hline
            US-4  & As a user I want the macine to handle difficulties automatically like maintenance so that I don't have to do anything         & 8          & M      \\ \hline
            US-5  & As a user I want to get displayed different data while the machine is brewing so that I can look at data                      & 5          & C      \\ \hline
            US-6  & As a user I want the machine to be reliable so that I am guaranteed the beers is good quality                                 & 6          & S      \\ \hline
            US-7  & As a user I want the machine to be efficient so that I don't have to wait too long to get a beer                              & 5          & C      \\ \hline
            US-8  & As a user I want the machine to be scalable so that I can add new beers to the machine                                        & 4          & C      \\ \hline
            US-9  & As a user I want the machine to have a quick repond time so that I don't have to wait for the machine to understand the input & 3          & W      \\ \hline
            US-10 & As a user I want the machine to be able to handle different amount of beers so that I can choose the amount I want            & 3          & W      \\ \hline

            \caption{User stories for the beer machine}
            \label{tab:user_stories}
        \end{longtable}
    \end{center}
\end{table}