
\section{Requirements}

Functional and non-functional requirements are listed in the table below. The requirements are based on the FURPS+ model, which is a model for classifying software quality attributes\cite{2}. The model is used to classify the requirements into different categories. The categories are functionality, usability, reliability, performance, and supportability.
To give a better overview of how the different non-functional requirements have been divided we have seperated each category of the FURPS+ model into different letters. Under the ID table, R stands for reliability, U stands for usability, S stands for scalability, P stands for performance, and lastly the + stands for the specific constraints consisting of Design constraints, Implementation constraints, Interface constraints, and Physical constraints. \newline

The requirements are also classified into the MoSCoW model, which is a model for prioritizing requirements. The categories are must have, should have, could have, and won't have. The requirements are listed in the table below. \newline

In the table below we have seperated each functional and non-functional requirement into the different categories in MoSCoW\cite{3}. M stands for must have, S stands for should have, C stands for could have and W stands for won't have.\newline

\subsection{Requirements using FURPS+}
\begin{center}
    \sloppy
    \begin{longtable}{|p{1cm}|p{4cm}|p{8.5cm}|c|}
        \hline
        ID & Name & Description & MoSCoW \\ \hline
        F-01   & Automated brewing process  & The beer machine needs to be able to automatically process the beer brewing process through a user-friendly interface.                                        & M \\ \hline
        F-02   & Recipe Management          & Users should be able to pick between different kinds of beer they want the machine to produce.                                                                & S \\ \hline                                                          
        F-03   & Alerts and Notifications   & Users should get alerts and notifications for the process if something fails, if the brewing of the beer was a success, or if the machine needs maintenance.  & C \\ \hline
        F-04   & Data collection            & The system needs to collect data from the different processes to ensure that we can improve the quality the brewing and minimize bad processing.              & S \\ \hline
        F-05   & Cleaning and maintenance   & The machine should contain a cleaning and maintenance program to ensure that the quality of the beer is in top.                                               & M \\ \hline
        F-06   & Low fail rate              & The machine must work at a speed that keeps the fail rate below 1\% if possible.                                                                              & M \\ \hline
        U-01   & User-friendly design       & We need a user-friendly design with clear instructions and easy navigation to help the user brew beers.         & M \\ \hline
        R-01   & Reliability                & The beer machine should operate consistently and with minimal failures to ensure beer quality.                                                                & M \\ \hline
        R-02   & Data backup                & The data needs to be stored in a database to ensure that no data will get lost from the brewing process.                                                      & S \\ \hline
        R-03   & Maintainability            & The design of the system should include well-documented code, apply modular design principles and meet industry-standard practices to ease future maintenance.& S \\ \hline 
        R-04   & Separation of Concerns     & The architecture of the system should separate the concerns through multiple deployments to enhance maintainability, scalability, and reliability.            & M \\ \hline
        P-01   & Performance                & It is important that the application has a quick response time and minimal failures during the brewing process.                                               & M \\ \hline
        P-02   & Response time              & The response time of the application needs to be quick to ensure that the user can navigate through the application without any delays.                       & M \\ \hline
        S-01   & Scalability                & The system should be scalable if new recipes or features are added to the machine.                                                                            & C \\ \hline
        S-02   & Future Development         & The program needs to be well structured in case the application needs to be optimized or improved further.                                                    & W \\ \hline

        \caption{Requirements using FURPS+ model}
        \label{tab:requirements}
    \end{longtable}
\end{center}

Table \ref{tab:requirements} shows the requirements for the beer machine. The requirements are based on the FURPS+ model, which is a model for classifying software quality attributes. For a better understanding we have seperated each requirement into their own category and given them a MoSCoW priority.







