
\section{Requirements}

Functional and non-functional\cite{example} requirements are listed in the table below. The requirements are based on the FURPS+ model, which is a model for classifying software quality attributes. The model is used to classify the requirements into different categories. The categories are functionality, usability, reliability, performance and supportability.
To give a better overview how the different non-functional requirements have been divided we have seperated each category of the FURPS+ model into different letters. Under the ID table R stands for reliability, U stands for usability, S stands for Scalability, P Stands for performance, and lastly the + for the spefic constraints in which contains Design constraints, Implementation constraints, Interface constraints and Physical constraints. \newline

The requirements are also classified into the MOSCOW model, which is a model for prioritizing requirements. The categories are must have, should have, could have and won't have. The requirements are listed in the table below. \newline

In the table below we have seperated each functional and non-functional requirements into the different categories in MOSCOW. M stands for must have, S stands for should have, C stands for could have and W stands for won't have.\newline

\subsection{Requirements using FURPS}
\begin{center}
    \sloppy
    \begin{longtable}{|p{1cm}|p{4cm}|p{8.5cm}|c|}
        \hline
        ID     & Name                       & Description                                                                                                                                                   & Moscow \\ \hline
        F-01   & Automated brewing process  & The beer machine needs to be able to automatically process the beer brewing process, through a user-friendly interface                                        & M \\ \hline
        F-02   & Recipe Management          & Users should be able to pick between different kinds of beer they want the machine to process.                                                                & S \\ \hline
        F-03   & Ingredients management     & The machine should automatically use the ingredients needed to brew a specific beer recipe.                                                                   & C \\ \hline
        F-04   & Alerts and Notifications   & Users should get alerts and notifications for the process if something fails or if the brewing of the beer was a success, or the machine needs maintenance.   & C \\ \hline
        F-05   & Data collection            & The system needs to collect data of the different processes to ensure we can improve the quality the brewing and minimize bad processing.                     & S \\ \hline
        F-06   & Cleaning and maintenance   & The machine should contain a cleaning and maintenance program to ensure that the quality of the beer is in top.                                               & M \\ \hline
        F-07   & Low failrete               & The machine must work in a speed that keeps the failrate below 1\% if possible                                                                                & M \\ \hline
        U-01   & User-friendly design       & To ensure that the process will be automated we need a user-friendly design with clear instructions and easy navigation to help the user brew beers.          & M \\ \hline
        R-01   & Reliability                & The beer machine should operate consistently and with minimal failures to ensure beer quality.                                                                & M \\ \hline
        R-02   & Data backup                & The data needs to be stores in a database to ensure no data will get lost from the brewing process.                                                           & S \\ \hline
        R-03   & Maintainability            & The design of the system should include well-documented code, modular design principles and meet industry-standard practices to ease future maintenance.      & S \\ \hline 
        R-04   & Separation of Concerns     & The architecture of the system should separate the concerns through multiple deployments to enhance maintainability, scalability and reliability.             & M \\ \hline
        P-01   & Performance                & It is important that the application has quick response times and minimal failures during the brewing process.                                                & M \\ \hline
        P-02   & Response time              & The response time of the application needs to be quick to ensure the user can navigate through the application without any delays.                            & M \\ \hline
        S-01   & Scalability                & The system should be able to scale if new recipes or features would be added to the machine.                                                                  & C \\ \hline
        S-02   & Future Development         & The program needs to be well structured, incase the application needs to be optimized or improved further                                                     & W \\ \hline
        

        \caption{Requirements using FURPS model}
        \label{tab:requirements}
    \end{longtable}
\end{center}

The Table \ref{tab:requirements} shows the requirements for the beer machine. The requirements are based on the FURPS+ model, which is a model for classifying software quality attributes. For a better understanding we have seperated each requierments into their own category and given them a Moscow priority.







