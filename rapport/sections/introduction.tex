\section{Introduction}

The goal of this project is to create a cyber-physical system, that includes a web interface, capable of communicating with a beer machine
through a OPC UA protocol. And a backend to ensure some automation of the brewing process.\newline

The web interface and backend is built using the knowledge of HTTP communication, HTML, JavaScript, and CSS from “Web Technologies”.
Further, it adheres to certain standards used in web development.

From “Industrial Cyber-Physical Systems” we got a basic understanding of how machines such as the beer machine operate,
and how to connect through a OPC UA protocol.

We have used the knowledge from “Calculus og Lineær Algebra” to properly handle data.

In “Operating systems and distributed systems” we learned how to properly seperate the program, and proper deploy aa product.

-----------------

The development of cyber-physical systems allows the physical and digital world to interact.
The systems can for example simplify traditional processes executed by humans.
In this project we create a cyber-physical system in which a web interface communicates with a beer machine through a OPC UA protocol.
We rely on knowledge gained in this semester's courses to properly develop and test the machine in accordance with the project requirements.

As a point of departure, we were provided with different instructions. These included/entailed
