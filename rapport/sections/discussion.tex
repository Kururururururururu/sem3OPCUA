\section{Discussion}

\subsection{Results}
\subsubsection{Achievements}
We have built a very robust backend, which requires very little input from the user to handle the entire brewing process. 
Automatically going into maintenance and refilling mode when required.
The backend has all the required implementations to scale up production to handle more than a single beer at a time.
We have connected our backend to the frontend, which enables the user to start the brewing process through the frontend, 
with a single click, and lets them see realtime update of the current inventory.
Whenever a beer is made, the connected database is also updated, storing which type of beer was produced, and when, which can later be used in Analytics.

\subsubsection{Limitations}
\subsubsubsection{Machine Speed Limitations}
The requirements we set for the failure rate,
is not possible to achieve on all the beer types,
as some require speeds outside the limitations set by the machine.

\subsubsection{Collaboration and Communication}
\subsubsubsection{Meetings}
Each Thursday we scheduled a meeting on campus between 10:00 and 14:00,
to go over any doubts about the code, or questions about writing the report.
This was done so everyone had some understanding of all the aspects 
of the codebase, and the report.


\subsubsubsection{Communication}
We had a group chat on Messenger, where we would write in case we were late,
or sick.
Besides that we had a Discord for sharing code, doing voice calls in case 
we did not show up at the meeting, or had questions about the code 
outside of the scheduled meeting times.

\subsubsection{Workflow}
For our workflow we used some aspects of the Scrum methodology, such as splitting the work into different sprints, and having a product backlog.
This helped splitting the tasks into manageable chunks, which could be completed in a single sprint.
Whenever a task in the sprint was completed, the code would be pushed to our git repository,
to make sure we all had a log of what was done, and had access to the code.

\subsection{Future Improvements}
We could have spent more time working on the frontend, as it is now, a lot of functionality is missing. 
While all the buttons are visually there,
few of them have actual functionality when clicked. \newline
Currently only the brew buttons work, and only for brewing a single beer at a time.
The start/stop buttons do nothing, and the analytics page does not get any data from the database.
\newline
So for the future development, these buttons should be hooked up to their respective functionalities in the backend,
and options for brewing multiple beers at a time could be added.