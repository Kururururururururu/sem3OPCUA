\section{Discussion}

\subsection{Results}
\subsubsection{Achievements}
We have managed to create a very robust backend, which needs very little input from the user, and handle the entire brewing process, 
and automatically going into maintenance and refilling mode when required.
The backend has all the required implementations to scale up production to handle more than a single beer at a time.
We have connected our backend to the frontend, which enables the user to start the brewing process through the frontend, 
with a single click, and letting them see realtime update of the current inventory.
Whenever a beer is made, the connected database is also updated, storing which type of beer was produced, and when, which can later be used in analytics.

\subsubsection{Limitations}
\subsubsubsection{Machine Speed Limitations}
The requirements we set for the failure rate,
is not possible to achieve on all the beer types,
as some require speeds outside the limitations set by the machine.

\subsubsection{Collaboration and Communication}
\subsubsubsection{Meetings}
Each thursday we scheduled a meeting on campus between 10:00 and 14:00,
to explain any doubts about code, or questions about writing the report.
This was done so everyone had some understanding of all the aspects 
of the codebase, and the report.


\subsubsubsection{Communication}
We had a group chat on messenger, where we would write in case we were late,
or sick.
Besides that we had a discord for sharing code, doing voice calls in case 
we didn't show up at the meeting, or had questions about the code 
outside of the scheduled meeting times.

\subsubsection{Workflow}
For our workflow we used some aspects of the scrum methodology, such as splitting the work into different sprints, and having a product backlog.
This helped splitting the tasks into manageable chunks, which could be completed in a single sprint.
Whenever a task in the sprint was completed, the code would be pushed to our git repository,
to make sure we all had a log of what was done, and had access to the code.

\subsection{Future Improvements}
\subsubsection{Frontend}
The frontend is not fully functional, as the only two features currently implemted, are the brewing process, and the inventory.
We could still have the stop/start buttons implemented, and the ability to brew more than a single beer at a time.
